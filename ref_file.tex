
	\begin{thebibliography}{1}
		
		
		\bibitem{r02}
		Brookmeyer, Ron, et al. ``Forecasting the global burden of Alzheimer’s disease." Alzheimer's \& dementia: the journal of the Alzheimer's Association 3.3 (2007): 186-191.
		
		\bibitem{r03}
		Musha, Toshimitsu, et al. ``EEG markers for characterizing anomalous activities of cerebral neurons in NAT (neuronal activity topography) method." IEEE Transactions on Biomedical Engineering 60.8 (2013): 2332-2338.
		
		\bibitem{r23}Golby, Alexandra, et al. ``Memory encoding in Alzheimer's disease: an fMRI study of explicit and implicit memory." Brain 128.4 (2005): 773-787.
		
		\bibitem{r04}Dennis, Emily L., and Paul M. Thompson. ``Functional brain connectivity using fMRI in aging and Alzheimer’s disease." Neuropsychology review 24.1 (2014): 49-62.
		
		\bibitem{r07}
		R. Graaf and K. Kevin. ``Methods and apparatus for
		compensating eld inhomogeneities in magnetic resonance"
		studies. US Patent No. 8035387, 2011.
		
		
		\bibitem{r33}N. Leonardi et al., “Principal components of functional connectivity: A
		new approach to study dynamic brain connectivity during rest,” NeuroImage,
		vol. 83, pp. 937–950, 2013.
		
		\bibitem{r34}Cherkassky, Vladimir L., et al. ``Functional connectivity in a baseline resting-state network in autism." Neuroreport 17.16 (2006): 1687-1690.
		
		\bibitem{r09}
		Stanley, Matthew Lawrence, et al. ``Defining nodes in complex brain networks." Frontiers in computational neuroscience 7 (2013): 169.
		
		\bibitem{r35}Du, Yuhui, Zening Fu, and Vince D. Calhoun. ``Classification and prediction of brain disorders using functional connectivity: promising but challenging." Frontiers in neuroscience 12 (2018).
		
		\bibitem{r36}de Vos, Frank, et al. ``A comprehensive analysis of resting state fMRI measures to classify individual patients with Alzheimer's disease." Neuroimage 167 (2018): 62-72.
		
		\bibitem{r37}Cuingnet, Rémi, et al. ``Automatic classification of patients with Alzheimer's disease from structural MRI: a comparison of ten methods using the ADNI database." neuroimage 56.2 (2011): 766-781.
		
		
			\bibitem{r38}Friston, Karl J. ``Functional and effective connectivity: a review." Brain connectivity 1.1 (2011): 13-36.
			
			
		\bibitem{r55}Leonardi, Nora, et al. ``Principal components of functional connectivity: a new approach to study dynamic brain connectivity during rest." NeuroImage 83 (2013): 937-950.
	
	%		\bibitem{r53}Supekar, Kaustubh, et al. "Network analysis of intrinsic functional brain connectivity in Alzheimer's disease." PLoS computational biology 4.6 (2008): e1000100.
	
	\bibitem{r54}Contreras, Joey A., et al. ``Resting state network modularity along the prodromal late onset Alzheimer's disease continuum." NeuroImage: Clinical 22 (2019): 101687.
	
	
	\bibitem{r60}Ahmadi, Soheil, and Mansoor Rezghi. ``A novel extension of Generalized Low-Rank Approximation of Matrices based on multiple-pairs of transformations." CoRR (2018).
	
	
	
	\bibitem{r10}
	Jie, Biao, et al. ``Integration of network topological and connectivity properties for neuroimaging classification." IEEE transactions on biomedical engineering 61.2 (2014): 576-589.
	
	%
	%		\bibitem{r11}
	%		Wee, Chong-Yaw, et al. "Resting-state multi-spectrum functional connectivity networks for identification of MCI patients." PloS one 7.5 (2012): e37828.
	
	%		\bibitem{r12}
	%		Tibshirani, Robert, et al. "Sparsity and smoothness via the fused lasso." Journal of the Royal Statistical Society: Series B (Statistical Methodology) 67.1 (2005): 91-108.
	%		
	%		
	%		
	%		\bibitem{r13}
	%		Wright, John, et al. "Robust face recognition via sparse representation." IEEE transactions on pattern analysis and machine intelligence 31.2 (2009): 210-227.
	
	\bibitem{r15}
	Huang, Shuai, et al. ``Learning brain connectivity of Alzheimer's disease by sparse inverse covariance estimation." NeuroImage 50.3 (2010): 935-949.
	
	
	\bibitem{r56}Leonardi, Nora, and Dimitri Van De Ville. ``On spurious and real fluctuations of dynamic functional connectivity during rest." Neuroimage 104 (2015): 430-436.
	
	
	%		\bibitem{r58}Barttfeld, Pablo, et al. "Signature of consciousness in the dynamics of resting-state brain activity." Proceedings of the National Academy of Sciences 112.3 (2015): 887-892.
	
	\bibitem{r59}Zalesky, Andrew, et al. ``Time-resolved resting-state brain networks." Proceedings of the National Academy of Sciences 111.28 (2014): 10341-10346.
	
	\bibitem{r57}
	Hindriks, Rikkert, et al. ``Can sliding-window correlations reveal dynamic functional connectivity in resting-state fMRI?." Neuroimage 127 (2016): 242-256.
		
	
	\bibitem{r64.5}Yan, Chaogan, and Yufeng Zang. "DPARSF: a MATLAB toolbox for" pipeline" data analysis of resting-state fMRI." Frontiers in systems neuroscience 4 (2010): 13.
	
	
	\bibitem{r64.7}Tzourio-Mazoyer, Nathalie, et al. "Automated anatomical labeling of activations in SPM using a macroscopic anatomical parcellation of the MNI MRI single-subject brain." Neuroimage 15.1 (2002): 273-289.
	
		
		\bibitem{n1}
	Leonardi, Nora, and Dimitri Van De Ville. ``Identifying network correlates of brain states using tensor decompositions of whole-brain dynamic functional connectivity." 2013 International Workshop on Pattern Recognition in Neuroimaging. IEEE, 2013.
	
	\bibitem{n2} Park, Sung Won. ``Multifactor analysis for fmri brain image classification by subject and motor task." Electrical and computer engineering technical report (2011).
	
	
	\bibitem{n3}Al-sharoa, Esraa, Mahmood Al-khassaweneh, and Selin Aviyente. ``Tensor Based Temporal and Multilayer Community Detection for Studying Brain Dynamics During Resting State fMRI." IEEE Transactions on Biomedical Engineering 66.3 (2018): 695-709.
	
	\bibitem{n4}Ozdemir, Alp, et al. ``Multi-scale higher order singular value decomposition (MS-HoSVD) for resting-state FMRI compression and analysis." 2017 IEEE International Conference on Acoustics, Speech and Signal Processing (ICASSP). IEEE, 2017.
	
	\bibitem{n5} He, Lifang, et al. ``Multi-way multi-level kernel modeling for neuroimaging classification." Proceedings of the IEEE Conference on Computer Vision and Pattern Recognition. 2017.
	
	\bibitem{n6} Ma, Guixiang, et al. ``Spatio-temporal tensor analysis for whole-brain fmri classification." Proceedings of the 2016 SIAM International Conference on Data Mining. Society for Industrial and Applied Mathematics, 2016.
	
	
	\bibitem{GLRAM}Ye, Jieping. ``Generalized low rank approximations of matrices." Machine Learning 61.1-3 (2005): 167-191.
	
	
		\bibitem{r64}Rezghi, Mansoor. ``A novel fast tensor-based preconditioner for image restoration." IEEE Transactions on Image Processing 26.9 (2017): 4499-4508.
		
		
	\bibitem{r14}
	Zhang, Jianjia, et al. ``Functional brain network classification with compact representation of SICE matrices." IEEE Transactions on Biomedical Engineering 62.6 (2015): 1623-1634.
	
	\bibitem{r14a}
	Chen, Xiaobo, et al. ``High‐order resting‐state functional connectivity network for MCI classification." Human brain mapping 37.9 (2016): 3282-3296.
%************************************************		
		
%		\bibitem{r21}
%		Nordberg, Agneta. "PET imaging of amyloid in Alzheimer's disease." The lancet neurology 3.9 (2004): 519-527.
%		
%		\bibitem{r22}
%		Jeong, Jaeseung. "EEG dynamics in patients with Alzheimer's disease." Clinical neurophysiology 115.7 (2004): 1490-1505.
		
		
		
		
		
		%		\bibitem{r04}
		%		Gould, R. L., et al. "Brain mechanisms of successful compensation during learning in Alzheimer disease." Neurology 67.6 (2006): 1011-1017.
		
		
		
		%		\bibitem{r05}
		%		Richiardi, Jonas, et al. "Classifying minimally disabled multiple sclerosis patients from resting state functional connectivity." Neuroimage 62.3 (2012): 2021-2033.
		
		%		\bibitem{r06}
		%		Yang, Xue, et al. "Evaluation of statistical inference on empirical resting state fMRI." IEEE Transactions on Biomedical Engineering 61.4 (2014): 1091-1099.
		
		
		
		
		
		%		\bibitem{r08}
		%		Zhang, Xiaowei, et al. "Resting-state whole-brain functional connectivity networks for mci classification using l2-regularized logistic regression." IEEE transactions on nanobioscience 14.2 (2015): 237-247.
		
		
		
		
		

		
		
		
		
%		\bibitem{r16}
%		Allen, Elena A., et al. "Tracking whole-brain connectivity dynamics in the resting state." Cerebral cortex 24.3 (2014): 663-676.
%		
%		\bibitem{r19}
%		Leonardi, Nora, et al. "Principal components of functional connectivity: a new approach to study dynamic brain connectivity during rest." NeuroImage 83 (2013): 937-950.
		
		
		
%		\bibitem{r61}Ng, Bernard, et al. "A novel sparse group Gaussian graphical model for functional connectivity estimation." International Conference on Information Processing in Medical Imaging. Springer, Berlin, Heidelberg, 2013.
		
		
		
%		\bibitem{r62}Colclough, Giles L., et al. "Multi-subject hierarchical inverse covariance modelling improves estimation of functional brain networks." NeuroImage 178 (2018): 370-384.
		
%		\bibitem{r63}	Foti, Nicholas J., and Emily B. Fox. "Statistical model-based approaches for functional connectivity analysis of neuroimaging data." Current opinion in neurobiology 55 (2019): 48-54.	
		
		
		
		
		
		
		
		
%		\bibitem{r49}V.Arsigny et al.,. (2006). Log-euclidean metrics for fast and simple calculus
%		on diffusion tensors. Magn. Reson. Med.. [Online]. 56(2), pp. 411–421.
%		Available: http://dx.doi.org/10.1002/mrm.20965
		
%		\bibitem{r50}S. Sra, “A new metric on the manifold of kernel matrices with application
%		to matrix geometric mean,” in Advances in Neural Information Processing
%		Systems 25, F. Pereira, C. J. C. Burges, L. Bottou, K. Q.Weinberger, Eds.,
%		New York, NY: Curran Associates, Inc., 2012, pp. 144–152.
		
%		\bibitem{r63.5}Chen, Xiaobo, et al. "High‐order resting‐state functional connectivity network for MCI classification." Human brain mapping 37.9 (2016): 3282-3296.
		
%		\bibitem{r51}Allen, Elena A., et al. "Tracking whole-brain connectivity dynamics in the resting state." Cerebral cortex 24.3 (2014): 663-676.
		
%		\bibitem{r51.5}Chang, Catie, and Gary H. Glover. "Time–frequency dynamics of resting-state brain connectivity measured with fMRI." Neuroimage 50.1 (2010): 81-98.
		
%		\bibitem{r52}Handwerker, Daniel A., et al. "Periodic changes in fMRI connectivity." Neuroimage 63.3 (2012): 1712-1719.
		
		
		
		
		
		
	
		
		
		
		
		
%		\bibitem{n}S. M. Smith et al., “Functional connectomics from resting-state fMRI,”
%		Trends Cognit. Sci., vol. 17, no. 12, pp. 666–682, 2013.
%		
%		\bibitem{nn2}N. Leonardi et al., “Principal components of functional connectivity: A
%		new approach to study dynamic brain connectivity during rest,” NeuroImage,
%		vol. 83, pp. 937–950, 2013.
		
%		\bibitem{nn3}C.-Y. Wee et al., “Constrained sparse functional connectivity networks
%		for MCI classification,” in Proc. Med. Image Comput. Comput.-Assisted
%		Intervention Conf., 2012, pp. 212–219.
		
		\bibitem{r65}Tan, Pang-Ning. ``Introduction to data mining." Pearson Education India, 2018.
		
		%%%*/*/*/*/*/*/*/*/*/ HERE HRRE
		
		%		\bibitem{r17}
		%		Damaraju, Eswar, et al. "Dynamic functional connectivity analysis reveals transient states of dysconnectivity in schizophrenia." NeuroImage: Clinical 5 (2014): 298-308.
		
		%		\bibitem{r18}
		%		Hutchison, R. Matthew, et al. "Dynamic functional connectivity: promise, issues, and interpretations." Neuroimage 80 (2013): 360-378.
		
		
		
		%		\bibitem{r20}
		%		Leonardi, Nora, et al. "Principal components of functional connectivity: a new approach to study dynamic brain connectivity during rest." NeuroImage 83 (2013): 937-950.
		
		
		
		\bibitem{r25}He, Yong, et al. ``Regional coherence changes in the early stages of Alzheimer’s disease: a combined structural and resting-state functional MRI study." Neuroimage 35.2 (2007): 488-500.
		
		
		
		\bibitem{r26}Bakkour, Akram, et al. ``The effects of aging and Alzheimer's disease on cerebral cortical anatomy: specificity and differential relationships with cognition." Neuroimage 76 (2013): 332-344.
		
		\bibitem{r27}Brewer, Alyssa A., and Brian Barton. ``Visual cortex in aging and Alzheimer's disease: changes in visual field maps and population receptive fields." Frontiers in psychology 5 (2014): 74.
		
		\bibitem{r28}Jacobsen, Jörn-Henrik, et al. ``Why musical memory can be preserved in advanced Alzheimer’s disease." Brain 138.8 (2015): 2438-2450.
		
		\bibitem{r29}Kosicek, Marko, and Silva Hecimovic. ``Phospholipids and Alzheimer’s disease: alterations, mechanisms and potential biomarkers." International journal of molecular sciences 14.1 (2013): 1310-1322.
		
		\bibitem{r30}Salvatore, Christian, et al. ``Magnetic resonance imaging biomarkers for the early diagnosis of Alzheimer's disease: a machine learning approach." Frontiers in neuroscience 9 (2015): 307.
		
		
		
		
		
		
		
		
		
		
		
		
		
		
		%		\bibitem{r39}Jones, David T., et al. "Non-stationarity in the “resting brain’s” modular architecture." PloS one 7.6 (2012): e39731.
		
		\bibitem{r40}Brickman, Adam M., et al. ``Reconsidering harbingers of dementia: progression of parietal lobe white matter hyperintensities predicts Alzheimer's disease incidence." Neurobiology of aging 36.1 (2015): 27-32.
		
		\bibitem{r41}De Reuck, J., et al. ``Topography of cortical microbleeds in Alzheimer’s disease with and without cerebral amyloid angiopathy: a post-mortem 7.0-tesla MRI Study." Aging and disease 6.6 (2015): 437.
		
		\bibitem{r42}Perani, Daniela, et al. ``The impact of bilingualism on brain reserve and metabolic connectivity in Alzheimer's dementia." Proceedings of the National Academy of Sciences 114.7 (2017): 1690-1695.
		
		\bibitem{r43}Watson, Rosie, et al. ``Subcortical volume changes in dementia with Lewy bodies and Alzheimer's disease. A comparison with healthy aging." International psychogeriatrics 28.4 (2016): 529-536.
		
		\bibitem{r44}Cai, Suping, et al. ``Changes in thalamic connectivity in the early and late stages of amnestic mild cognitive impairment: a resting-state functional magnetic resonance study from ADNI." PloS one 10.2 (2015): e0115573.
		
		\bibitem{r45}Ortner, Marion, et al. ``Progressively Disrupted intrinsic Functional connectivity of Basolateral amygdala in Very early alzheimer’s Disease." Frontiers in neurology 7 (2016): 132.
		
		\bibitem{r46}Steketee, Rebecca ME, et al. ``Early-stage differentiation between presenile Alzheimer’s disease and frontotemporal dementia using arterial spin labeling MRI." European radiology 26.1 (2016): 244-253.
		
		\bibitem{r47}Sanz-Arigita, Ernesto J., et al. ``Loss of ‘small-world’networks in Alzheimer's disease: graph analysis of FMRI resting-state functional connectivity." PloS one 5.11 (2010): e13788.
		
		\bibitem{r48}Zhang, Daoqiang, et al. ``Multimodal classification of Alzheimer's disease and mild cognitive impairment." Neuroimage 55.3 (2011): 856-867.
		
		
		\bibitem{r32}Jacobs, Heidi IL, et al. ``The cerebellum in Alzheimer’s disease: evaluating its role in cognitive decline." Brain 141.1 (2017): 37-47.
		
	
		
	\end{thebibliography}
	